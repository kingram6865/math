\documentclass[10pt,letterpaper]{article}
\usepackage{xcolor}
\usepackage[utf8]{inputenc}
\usepackage{amsmath}
\usepackage{amsfonts}
\usepackage{amssymb}
\usepackage{graphicx}
\usepackage[left=2cm,right=2cm,top=2cm,bottom=2cm]{geometry}
\begin{document}
\textbf{Section 1.3 - Point-of-Division Formulas}
\newline \textbf{Problems}
\newline \textbf{A}
\newline \textit{In Problems 1-6, find the point $P$ such that $\overline{AP}/\overline{AB} = r$}
\newline 1. $A=(3,4), B=(7,0) r=1/4$
\newline 2. $A=(4,-2), B=(-2,-5) r=2/3$
\newline 3. $A=(5, -1), B=(-4,-5) r=1/5$
\newline 4. $A=(2,4), B=(-5,2) r=2/5$
\newline 5. $A=(-4,1), B=(3,8) r=3$
\newline 6. $A=(-6,2), B=(4,4) r=5/2$\\
\newline \textit{In Problems 7-10, find the midpoint of the segment $AB$.}
\newline 7. $A=(5,-2), B=(-1,4)$
\newline 8. $A=(-3,3), B=(1,5)$
\newline 9. $A=(4,-1), B=(3,3)$
\newline 10. $A=(-1,4), B=(0,2)$\\
\newline \textbf{B}
\newline 11. $If A=(3,5), P=(6,2) and AP/AB=1/3, find B$
\newline 12. $If P=(4,7), B=(2,-1) and AP/AB=2/5, find A$
\newline 13. $If P=(2,-5), B=(4,-3) and AP/AB=1/2, find A$
\newline 14. $If A=(3,3), P=(5,2) and AP/AB=3/5, find B$\\
\newline \textit{In Problems 15-18, find the point $P$ between $A$ and $B$ such that $AB$ is divided in the given ratio.}
\newline 15. $A=(5,-3), B=(-1,6), AP/AB=1/2$
\newline 16. $A=(-1,-3), B=(-8,11), AP/AB=3/4$
\newline 17. $A=(2,-1), B=(4,5), AP/AB=2/3$
\newline 18. $A=(5,8), B=(2,-1), AP/AB=5/1$\\
\newline \textbf{C}
\newline 19. If $P=(4,-1)$ is the midpoint of the segment $AB$, where $A=(2,5)$, find $B$.
\newline 20. Find the center and radius of the circle circumscribed about the right triangle with vertices $(1,1)$, $(1,4)$ and $(7,4)$.
\newline 21. Find the center and radius of the circle circumscribed about the triangle with vertices $(5,2)$, $(0,4)$ and $(-1, -1)$ (See Problem 26).
\newline 22. Prove analytically that the diagonals of a parallelogram bisect each other.
\newline 23. Find the point of intersection of the diagonals of the paralellogram with vertices $(1,1)$, $(4,1)$, $(3,-2)$ and $(0,-2)$.
\newline 24. Prove analytically that the midpoint of the hypotenuse of a right triangle is equidistant from the three vertices.
\newline 25. Prove analytically that the vertex and the midpoints of the three sides of an isoceles triangle are the vertices of a rhombus.
\newline 26. Prove analytically that the meridians of a triangle are concurrent at a point two-thirds of the way from each vertex to the midpoint of the opposite side.
\newline 27. The point $(1,4)$ is at a distance 5 from the midpoint of the segment joining $(3, -2)$ and $(x,4)$. Find $x$.
\newline 28. The midpoints of the sides of a triangle are $(-1,3)$, $(1,-2)$, and $(5,-3)$. Find the vertices.
\newline 29. Three vertices of a parallelogram are $(2,5)$, $(-7,1)$, and $(4,-6)$. Find the fourth vertex. [Hint: There is more than one solution. Sketch all possible parallelograms using the three given vertices.]
\newline 30. Show that if a triangle has vertices $(x_1,y_1)$, $(x_2,y_2)$ and $(x_3,y_3)$, then the point of intersection of its meridians is 
\[ \left(\frac{x_1 + x_2 + x_3}{3}, \frac{y_1 + y_2 + y_3}{3}\right) \]
\newline 31. Prove analytically that the sum of the squares of the four sides of a parallelogram is equal to the sum of the squares of the two diagonals.
\end{document}