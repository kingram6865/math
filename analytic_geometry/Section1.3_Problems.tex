\documentclass[10pt,letterpaper]{article}
\usepackage{xcolor}
\usepackage[utf8]{inputenc}
\usepackage{amsmath}
\usepackage{amsfonts}
\usepackage{amssymb}
\usepackage{graphicx}
\usepackage[left=2cm,right=2cm,top=2cm,bottom=2cm]{geometry}
\begin{document}

\textbf{Section 1.3 - Point-of-Division Formulas}
\medskip

\textbf{Problems}

\medskip
\textbf{A}
\medskip

\textit{In Problems 1-6, find the point $P$ such that $\overline{AP}/\overline{AB} = r$}
\medskip

Coordinates of $P = (x,y)$ where 
\[ x=x_1 + r(x_2 - x_1)  \]
\[ y=y_1 + r(y_2 - y_1)  \]
\medskip

1. $A=(3,4), B=(7,0)\  r=1/4$

\medskip
\fcolorbox{blue}{white} {\textbf{{\color{blue}Solution}:}}

\medskip
$x =3 + \frac{1}{4} (7-3) = 3 + \frac{1}{4} (4) = 4$

\medskip
$y = 4 + \frac{1}{4}(0-4) = 4 + \frac{1}{4}(-4) = 4 + (-1) = 4 - 1 = 3$

\medskip
$\therefore P(4,3)$

\noindent\rule{15cm}{0.4pt}

\medskip
2. $A=(4,-2), B=(-2,-5)\  r=2/3$

\medskip
\fcolorbox{blue}{white} {\textbf{{\color{blue}Solution}:}}

\smallskip
$x=4+\frac{2}{3}(-2-4)=4+\frac{2}{3}(-6)=4+(-4)=4-4=0$

\smallskip
$y=-2+\frac{2}{3}(-5-(-2))=2+\frac{2}{3}(-5+2)=2+\frac{2}{3}(-3)=2+2(-1)=$

\smallskip
$\therefore P(0,-4)$

\noindent\rule{15cm}{0.4pt}

\medskip
3. $A=(5, -1), B=(-4,-5)\  r=1/5$

\medskip
\fcolorbox{blue}{white} {\textbf{{\color{blue}Solution}:}}

\smallskip
$x=5+\frac{1}{5}(-4-5)=5+\frac{1}{5}(-9)=5-\frac{9}{5}=\frac{25}{5}-\frac{9}{5}=\frac{16}{5}$

\smallskip
$y=-1+\frac{1}{5}(-5-(-1))=-1+\frac{1}{5}(-4)=-1-\frac{4}{5}=-\frac{5}{5}-\frac{4}{5}=-\frac{9}{5}$

\smallskip
$\therefore P(\frac{16}{5},\frac{-9}{5})$

\noindent\rule{15cm}{0.4pt}

\medskip
4. $A=(2,4), B=(-5,2)\  r=2/5$

\medskip
\fcolorbox{blue}{white} {\textbf{{\color{blue}Solution}:}}

\smallskip
$x=2+\frac{2}{5}(-5-2)=2+\frac{2}{5}(-7)=2+2(-2)=2+\frac{-14}{5}=\frac{10}{5}-\frac{14}{5}=-\frac{4}{5}$

\smallskip
$y=4+\frac{2}{5}(2-4))=4+\frac{2}{5}(-2)=4+(-\frac{4}{5})=4-\frac{4}{5}=\frac{20}{5}-\frac{4}{5}=\frac{16}{5}$

\smallskip
$\therefore P=(-\frac{4}{5},\frac{16}{5})$

\noindent\rule{15cm}{0.4pt}

\medskip
5. $A=(-4,1), B=(3,8)\  r=3$

\medskip
\fcolorbox{blue}{white} {\textbf{{\color{blue}Solution}:}}

\smallskip
$x=-4+3(3-(-4))=-4+3(3+4=-4+(3 \cdot 4))=-4+21=17$

\smallskip
$y=3+3(8-1)=3+(3\cdot7)=3+21=24$

\smallskip
$\therefore \color{green}P=(17,24)$

\noindent\rule{15cm}{0.4pt}
\pagebreak

6. $A=(-6,2), B=(4,4)\  r=5/2$

\medskip
\fcolorbox{blue}{white} {\textbf{{\color{blue}Solution}:}}

\smallskip
$x=-6+\frac{5}{2}(4-(-6))=-6+\frac{5}{2}(4+6)=-6+\frac{5}{2}(10)=-6+(5\cdot5)=-6+25=19$

\smallskip
$y=2+\frac{5}{2}(4-2)=2+\frac{5}{2}(2)=2+5=7$

\smallskip
$\therefore P=(19,7)$

\noindent\rule{15cm}{0.4pt}

\medskip
\textit{In Problems 7-10, find the midpoint of the segment $\overline{AB}$.}
\medskip

$P_{midpoint} = (x,y)$ where 
\[ x=\frac{x_1 + x_2}{2},\  y=\frac{y_1 + y_2}{2} \]
\medskip

\noindent\rule{15cm}{0.4pt}

\medskip
7. $A=(5,-2), B=(-1,4)$

\medskip
\fcolorbox{blue}{white} {\textbf{{\color{blue}Solution}:}}

\smallskip
$x=\frac{5+(-1)}{2}=\frac{-4}{2}=-2$

\smallskip
$y=\frac{4-2}{2}=\frac{2}{2}=1$

\smallskip
$\therefore P_{midpoint}=(-2,1)$

\noindent\rule{15cm}{0.4pt}

\medskip
8. $A=(-3,3), B=(1,5)$

\medskip
\fcolorbox{blue}{white} {\textbf{{\color{blue}Solution}:}}

\smallskip
$x=\frac{-3+1}{2}=\frac{-2}{2}=-1$

\smallskip
$y=\frac{3+5}{2}=\frac{8}{2}=4$

\smallskip
$\therefore P_{midpoint}=(-1,4)$

\noindent\rule{15cm}{0.4pt}

\medskip
9. $A=(4,-1), B=(3,3)$

\medskip
\fcolorbox{blue}{white} {\textbf{{\color{blue}Solution}:}}

\smallskip
$x=\frac{4+3}{2}$

\smallskip
$y=\frac{-1+3}{2}$

\smallskip
$\therefore P_{midpoint}=(,)$

\noindent\rule{15cm}{0.4pt}

\medskip
10. $A=(-1,4), B=(0,2)$

\medskip
\fcolorbox{blue}{white} {\textbf{{\color{blue}Solution}:}}

\medskip
$x=\frac{-1+0}{2}=\frac{-1}{2}$

\medskip
$y=\frac{4+2}{2}=\frac{6}{2}=3$

\medskip
$\therefore P_{midpoint}=(\frac{-1}{2},3)$

\noindent\rule{15cm}{0.4pt}

\pagebreak
\textbf{B}

\medskip
\textit{Recall that $\frac{\overline{AP}}{\overline{AB}} = r$}
\textit{and this implies $\overline{AP} = r\cdot\overline{AB}$}

\medskip
\textit{Furthermore }
\[x=x_1+r(x_2-x_1)\]
\[y=y_1+r(y_2-y_1)\]

\medskip

\medskip
11. $If A=(3,5), P=(6,2)$ and $\overline{AP}/\overline{AB}=1/3$, find B

\medskip
\fcolorbox{blue}{white} {\textbf{{\color{blue}Solution}:}}

\medskip
$B=(x_1,y_1)=?$

\medskip
$6=3+\frac{1}{3}(x_1-3)=3+\frac{1}{3}x_1-\frac{1}{3}(3)=3-\frac{3}{3}+\frac{1}{3}x_1=3-1+\frac{1}{3}x_1$

\medskip
$6=2+\frac{1}{3}x_1\implies6-2=\frac{1}{3}x_1\implies3(6-2)=3(\frac{1}{3}x_1)\implies3\cdot4=x_1$

\medskip
$\therefore x_1=12$

\medskip
$2=5+\frac{1}{3}(y_1-5)=5+\frac{1}{3}y_1-\frac{1}{3}(5)=5\frac{5}{3}+\frac{1}{3}y_1=\frac{15}{3}-\frac{5}{3}+\frac{1}{3}y_1$

\medskip
$2=\frac{10}{3}+\frac{1}{3}y_1 \implies 2-\frac{10}{3}=\frac{1}{3}y_1 \implies \frac{6}{3}-\frac{10}{3}=\frac{1}{3}y_1 \implies -\frac{4}{3}=\frac{1}{3}y_1$

\medskip
$\therefore y_1=-4$

\medskip
So $B = (12, -4)$

\noindent\rule{15cm}{0.4pt}

\medskip
12. $If P=(4,7), B=(2,-1)$ and $\overline{AP}/\overline{AB}=2/5$, find A

\medskip
\fcolorbox{blue}{white} {\textbf{{\color{blue}Solution}:}}

\medskip
$A=(x_1,y_1)=?$

\medskip
$4 = 2+\frac{2}{5}(2-x_1)=2+\frac{2}{5}(2)-\frac{2}{5}x_1=2+\frac{4}{5}-\frac{2}{5}x_1=\frac{10}{5}+\frac{4}{5}-\frac{2}{5}x_1$

\medskip
$4=\frac{14}{5}-\frac{2}{5}x_1 \implies 4=\frac{14}{5}-\frac{2}{5}x_1 \implies 4-\frac{14}{5}=-\frac{2}{5}x_1 \implies -(\frac{5}{2})(\frac{20}{5}-\frac{14}{5})=x_1$

\medskip
$\therefore x_1=-(\frac{5}{2})(\frac{6}{5})=-3$

\medskip
$7=-1+\frac{2}{5}(-1-y_1)=-1-\frac{2}{5}-\frac{2}{5}y_1=-\frac{5}{5}-\frac{2}{5}-\frac{2}{5}y_1=-\frac{7}{5}-\frac{2}{5}y_1$

\medskip
$7+\frac{7}{5}=-\frac{2}{5}y_1 \implies \frac{35}{5}+\frac{7}{5}=-\frac{2}{5}y_1 \implies \frac{42}{5}=-\frac{2}{5}y_1 \implies (-\frac{5}{2})(\frac{42}{5})=y_1$

\medskip
$\therefore y_1=-21$

\medskip
So $A=(-3,-21)$

\noindent\rule{15cm}{0.4pt}

\medskip
13. $If P=(2,-5), B=(4,-3)$ and $\overline{AP}/\overline{AB}=1/2$, find A

\medskip
\fcolorbox{blue}{white} {\textbf{{\color{blue}Solution}:}}

\medskip
$2=4+\frac{1}{2}(4-x_1)=4+\frac{1}{2}(4)-\frac{1}{2}x_1=4+2-\frac{1}{2}x_1=6-\frac{1}{2}x_1$

\medskip
$2-6=-\frac{1}{2}x_1 \implies -4=-\frac{1}{2}x_1 \implies (-2)(-4)=x_1$

\medskip
$\therefore x_1=8$

\medskip
$-5=-3+\frac{1}{2}(-3-y_1)=-3-\frac{1}{2}(3)-\frac{1}{2}y_1=-\frac{6}{2}-\frac{3}{2}-\frac{1}{2}y_1=-\frac{9}{2}-\frac{1}{2}y_1$

\medskip
$-5+\frac{9}{2}=-\frac{1}{2}y_1 \implies -\frac{10}{2}+\frac{9}{2}=-\frac{1}{2}y_1 \implies -\frac{1}{2}=-\frac{1}{2}y_1 \implies 1=y_1$

\medskip
$\therefore y_1=1$

\medskip
So $A=(8,1)$

\noindent\rule{15cm}{0.4pt}

\pagebreak

14. $If A=(3,3), P=(5,2)$ and $\overline{AP}/\overline{AB}=3/5$, find B

\medskip
\fcolorbox{blue}{white} {\textbf{{\color{blue}Solution}:}}

\medskip
$5=3+\frac{3}{5}(x_2-3)=\frac{15}{5}+\frac{3}{5}x_2-(\frac{3}{5})(3)=\frac{15}{5}-\frac{9}{5}+\frac{3}{5}x_2=\frac{6}{5}+\frac{3}{5}x_2$

\medskip
$5-\frac{6}{5}=\frac{3}{5}x_2 \implies \frac{25}{5}-\frac{6}{5}=\frac{3}{5}x_2 \implies \frac{19}{5}=\frac{3}{5}x_2 \implies (\frac{5}{3})(\frac{19}{5})=x_2$

\medskip
$\therefore x_2=\frac{19}{3}$

\medskip
$2=3+\frac{3}{5}(y_1-3)=\frac{15}{5}+\frac{3}{5}y_1-(3)(\frac{3}{5})=\frac{15}{5}-\frac{9}{5}+\frac{3}{5}y_1=\frac{6}{5}+\frac{3}{5}y_1$

\medskip
$2=\frac{6}{5}+\frac{3}{5}y_1 \implies \frac{10}{5}=\frac{6}{5}+\frac{3}{5}y_1 \implies \frac{10}{5}-\frac{6}{5}=\frac{3}{5}y_1 \implies \frac{4}{5}=\frac{3}{5}y_1 \implies (\frac{5}{3})(\frac{4}{5})=y_1$

\medskip
$\therefore y_1=\frac{4}{3}$

\medskip
So $B=(\frac{19}{3},\frac{4}{3})$

\noindent\rule{15cm}{0.4pt}

\medskip
\textit{In Problems 15-18, find the point $P$ between $A$ and $B$ such that $AB$ is divided in the given ratio.}
\medskip

\medskip
The midpoint formula is \[ x = \frac{x_1+x_2}{2} \qquad y=\frac{y_1+y_2}{2} \]


\noindent\rule{15cm}{0.4pt}

\medskip
15. $A=(5,-3), B=(-1,6)$, $\overline{AP}/\overline{PB}=1/2$

\medskip
\fcolorbox{blue}{white} {\textbf{{\color{blue}Solution}:}}

\medskip



\noindent\rule{15cm}{0.4pt}

\medskip
16. $A=(-1,-3), B=(-8,11)$, $\overline{AP}/\overline{PB}=3/4$

\medskip
\fcolorbox{blue}{white} {\textbf{{\color{blue}Solution}:}}

\noindent\rule{15cm}{0.4pt}

\medskip
17. $A=(2,-1), B=(4,5)$, $\overline{AP}/\overline{PB}=2/3$

\medskip
\fcolorbox{blue}{white} {\textbf{{\color{blue}Solution}:}}

\noindent\rule{15cm}{0.4pt}

\medskip
18. $A=(5,8), B=(2,-1)$, $\overline{AP}/\overline{PB}=5/1$

\medskip
\fcolorbox{blue}{white} {\textbf{{\color{blue}Solution}:}}


\noindent\rule{15cm}{0.4pt}

\medskip
\textbf{C}
\medskip

\noindent\rule{15cm}{0.4pt}

\medskip
19. If $P=(4,-1)$ is the midpoint of the segment $AB$, where $A=(2,5)$, find $B$.

\medskip
\fcolorbox{blue}{white} {\textbf{{\color{blue}Solution}:}}

\noindent\rule{15cm}{0.4pt}

\medskip
20. Find the center and radius of the circle circumscribed about the right triangle with vertices $(1,1)$, $(1,4)$ and $(7,4)$.

\medskip
\fcolorbox{blue}{white} {\textbf{{\color{blue}Solution}:}}

\noindent\rule{15cm}{0.4pt}

\medskip
21. Find the center and radius of the circle circumscribed about the triangle with vertices $(5,2)$, $(0,4)$ and $(-1, -1)$ (See Problem 26).

\medskip
\fcolorbox{blue}{white} {\textbf{{\color{blue}Solution}:}}

\noindent\rule{15cm}{0.4pt}

\medskip
22. Prove analytically that the diagonals of a parallelogram bisect each other.

\medskip
\fcolorbox{blue}{white} {\textbf{{\color{blue}Solution}:}}

\noindent\rule{15cm}{0.4pt}

\medskip
23. Find the point of intersection of the diagonals of the paralellogram with vertices $(1,1)$, $(4,1)$, $(3,-2)$ and $(0,-2)$.

\medskip
\fcolorbox{blue}{white} {\textbf{{\color{blue}Solution}:}}

\noindent\rule{15cm}{0.4pt}

\medskip
24. Prove analytically that the midpoint of the hypotenuse of a right triangle is equidistant from the three vertices.

\medskip
\fcolorbox{blue}{white} {\textbf{{\color{blue}Solution}:}}

\noindent\rule{15cm}{0.4pt}

\medskip
25. Prove analytically that the vertex and the midpoints of the three sides of an isoceles triangle are the vertices of a rhombus.

\medskip
\fcolorbox{blue}{white} {\textbf{{\color{blue}Solution}:}}

\noindent\rule{15cm}{0.4pt}

\medskip
26. Prove analytically that the meridians of a triangle are concurrent at a point two-thirds of the way from each vertex to the midpoint of the opposite side.

\medskip
\fcolorbox{blue}{white} {\textbf{{\color{blue}Solution}:}}

\noindent\rule{15cm}{0.4pt}

\medskip
27. The point $(1,4)$ is at a distance 5 from the midpoint of the segment joining $(3, -2)$ and $(x,4)$. Find $x$.

\medskip
\fcolorbox{blue}{white} {\textbf{{\color{blue}Solution}:}}

\noindent\rule{15cm}{0.4pt}

\medskip
28. The midpoints of the sides of a triangle are $(-1,3)$, $(1,-2)$, and $(5,-3)$. Find the vertices.

\medskip
\fcolorbox{blue}{white} {\textbf{{\color{blue}Solution}:}}

\noindent\rule{15cm}{0.4pt}

\medskip
29. Three vertices of a parallelogram are $(2,5)$, $(-7,1)$, and $(4,-6)$. Find the fourth vertex. [Hint: There is more than one \color{violet}solution\color{black}. Sketch all possible parallelograms using the three given vertices.]

\medskip
\fcolorbox{blue}{white} {\textbf{{\color{blue}Solution}:}}

\noindent\rule{15cm}{0.4pt}

\medskip
30. Show that if a triangle has vertices $(x_1,y_1)$, $(x_2,y_2)$ and $(x_3,y_3)$, then the point of intersection of its meridians is 
\[ \left(\frac{x_1 + x_2 + x_3}{3}, \frac{y_1 + y_2 + y_3}{3}\right) \]

\medskip
\fcolorbox{blue}{white} {\textbf{{\color{blue}Solution}:}}

\noindent\rule{15cm}{0.4pt}

\medskip
31. Prove analytically that the sum of the squares of the four sides of a parallelogram is equal to the sum of the squares of the two diagonals.

\medskip
\fcolorbox{blue}{white} {\textbf{{\color{blue}Solution}:}}

\end{document}