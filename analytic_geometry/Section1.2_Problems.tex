\documentclass[10pt,letterpaper]{article}
\usepackage{xcolor}
\usepackage[utf8]{inputenc}
\usepackage{amsmath}
\usepackage{amsfonts}
\usepackage{amssymb}
\usepackage{graphicx}
\usepackage[left=2cm,right=2cm,top=2cm,bottom=2cm]{geometry}
\definecolor{amethyst}{rgb}{0.6, 0.4, 0.8}

\begin{document}

\textbf{Section 1.2 - Distance Formula}

\textbf{Problems}

\textbf{A}

\textit{In problems 1-8, find the distance between the given points}

1. $A=(1,-3), B=(2,5)$ $\color{blue} => x_1=1, x_2=2$, $\color{green}y_1=-3, y_2=5 $

\textbf{\underline {Solution}:}

$\overline{AB}=\sqrt{(2-1)^2+(5-(-3))^2}=\sqrt{(1)^2+(5+3)^2}=\sqrt{1^2+8^2}=\sqrt{1+64}=\sqrt{65}$

2. $A=(4,13), B=(-1,5)$ $\color{blue} => x_1 =4, x_2=-1$, $\color{green}y_1=13, y_2=5 $

\textbf{\underline {Solution}:}

$\overline{AB}=\sqrt{(-1-4)^2+(13-5)^2}=\sqrt{(-5)^2+(8)^2}=\sqrt{25+64}=\sqrt{89}$

3. $A=(3, -2), B=(3, -4)$ $\color{blue} => x_1 =3, x_2=-2$, $\color{green}y_1=3, y_2=-4 $

\textbf{\underline {Solution}:}

$\overline{AB}=\sqrt{(-2-3)^2 + (-4-3)^2}=\sqrt{(-5)^2+(-7)^2}=\sqrt{25+49}=\sqrt{74}$

4. $A=(-5, 1), B=(0, -10)$ $\color{blue} => x_1=-5, x_2=0$, $\color{green}y_1=1, y_2=-10 $

\textbf{\underline {Solution}:}

$\overline{AB}=\sqrt{(0-(-5))^2+(-10-1)^2}=\sqrt{(0+5)^2+(-11)^2}=\sqrt{5^2+(-11)^2}=\sqrt{25+121}=\sqrt{146}$

5. $A=(\frac{1}{2}, \frac{3}{2}), B=(\frac{-5}{2}, 2)$ $\color{blue} => x_1=\frac{1}{2}, x_2=\frac{-5}{2}$, $\color{green}y_1=\frac{3}{2}, y_2=2 $

\textbf{\underline {Solution}:}

$\overline{AB}=\sqrt{(\frac{-5}{2}-\frac{1}{2})^2+(2-\frac{3}{2})^2}=\sqrt{(\frac{-6}{2})^2+(\frac{4}{2}-\frac{3}{2})^2}=\sqrt{(-3)^2+(\frac{1}{2})^2}=\sqrt{9+\frac{1}{4}}=\sqrt{\frac{36}{4}+\frac{1}{4}}=\sqrt{\frac{37}{4}}=\frac{\sqrt{37}}{2}$

6. $A=(\frac{2}{3}, \frac{1}{3}), B=(\frac{-4}{3}, \frac{4}{3})$ $\color{blue} => x_1=\frac{2}{3}, x_2=\frac{-4}{3}$, $\color{green}y_1=\frac{1}{3}, y_2=\frac{4}{3} $

\textbf{\underline {Solution}:}

$\overline{AB}=\sqrt{(\frac{-4}{3}-\frac{2}{3})^2+(\frac{4}{3}-\frac{1}{3})^2}=\sqrt{(\frac{-6}{3})^2+(\frac{3}{3})^2}=\sqrt{(-2)^2+(1)^2}=\sqrt{4+1}=\sqrt{5}$

7. $A=(\sqrt{2}, 1), B=(2\sqrt{2}, 3)$ $\color{blue} => x_1=\sqrt{2}, x_2=2$, $\color{green}y_1=1, y_2=\sqrt{2} $

\textbf{\underline {Solution}:}

$\overline{AB}=\sqrt{(2\sqrt{2}-\sqrt{2})^2+(3-1)^2}=\sqrt{(\sqrt{2})^2+(2)^2}=\sqrt{2+4}=\sqrt{6}$

8. $A=(\sqrt{3}, -\sqrt{2}), B=(-3\sqrt{3}, \sqrt{2})$ $\color{blue} => x_1=\sqrt{3}, x_2=-3\sqrt{3}$, $\color{green}y_1=13, y_2= $

\textbf{\underline {Solution}:}

$\overline{AB}=\sqrt{(-3\sqrt{3}-\sqrt{3})^2+(\sqrt{2}-(-\sqrt{2}))^2}=\sqrt{(-4\sqrt{3})^2+(\sqrt{2}+\sqrt{2})^2}$

$=\sqrt{(-4\sqrt{3})^2+(2\sqrt{2})}=\sqrt{(-4)^2(\sqrt{3})^2+(2)^2(\sqrt{2})^2}=\sqrt{(16)(3)+(4)(2)}=\sqrt{48+8}=\sqrt{56}$

\textit{In Problems 9-14, determine whether the three given points are collinear}

\begin{quote}
{}[Given 3 points A, B, C, these points are collinear if $\overline{AC}=\overline{AB}+\overline{BC}$ or $\overline{AC}=\overline{BC}-\overline{AB}$]
\end{quote}

9. $A=(2, 1), B=(4, 3), C=(-1, -2)$

\textbf{\underline {Solution}:}

$\overline{AB} =\sqrt{(4-2)^2+(3-1)^2}=\sqrt{(2)^2+(2)^2}=\sqrt{4+4}=\sqrt{8}=2\sqrt{2}$

$\overline{BC} =\sqrt{(-1-4)^2+(-2-3)^2}=\sqrt{(-5)^2+(-5)^2}=\sqrt{25+25}=\sqrt{50}=\sqrt{25}\sqrt{2}=5\sqrt{2}$

$\overline{AC} =\sqrt{(-1-2))^2+(-2-1)^2}=\sqrt{(-3)^2+(-3)^2}=\sqrt{9+9}=\sqrt{18}=\sqrt{9}\sqrt{2}=3\sqrt{2}$

10. $A=(3, 2), B=(4, 6), C=(0, -8)$

\textbf{\underline {Solution}:}

$\overline{AB} =\sqrt{(4-3)^2+(6-2)^2}=\sqrt{(1)^2+(4)^2}=\sqrt{1+16}=\sqrt{17}$

$\overline{BC} =\sqrt{(0-4)^2+(-8-6)^2}=\sqrt{(-4)^2+(-14)^2}=\sqrt{16+196}=\sqrt{212}=\sqrt{4}\sqrt{53}=2\sqrt{53}$

$\overline{AC} =\sqrt{(0-3)^2+(-8-2)^2}=\sqrt{(-3)^2+(-10)^2}=\sqrt{9+100}=\sqrt{109}$


11. A=$(-2, 3), B=(7, -2), C=(2, 5)$

\textbf{\underline {Solution}:}

$\overline{AB} =\sqrt{(7-(-2))^2+(-2-3)^2}=\sqrt{(7+2)^2+(-5)^2}=\sqrt{9^2+25}=\sqrt{81+25}=\sqrt{106}=\sqrt{4}\sqrt{26}=2\sqrt{26}$

$\overline{BC} =\sqrt{(2-7)^2+(5-(-2))^2}=\sqrt{(-5)^2+(5+2)^2}=\sqrt{25+7^2}=\sqrt{25+49}=\sqrt{74}$

$\overline{AC} =\sqrt{(2-(-2))^2+(5-3)^2}=\sqrt{(2+2)^2+(2)^2}=\sqrt{4^2+4}=\sqrt{16+4}=\sqrt{20}=\sqrt{4}\sqrt{5}=2\sqrt{5}$


12. A=$(2, -1), B=(-1, 4), C=(5, -6)$

\textbf{\underline {Solution}:}

$\overline{AB} =\sqrt{(-1-2)^2+(4-(-1))^2}=\sqrt{(-3)^2+(4+1)^2}=\sqrt{9+(5)^2}=\sqrt{9+25}=\sqrt{34}$

$\overline{BC} =\sqrt{(5-(-1))^2+(-6-4)^2}=\sqrt{(5+1)^2+(-10)^2}=\sqrt{(6)^2+100}==\sqrt{36+100}=\sqrt{136}=4\sqrt{34}$

$\overline{AC} =\sqrt{(5-2)^2+(-6-(-1))^2}=\sqrt{(3)^2+(-6+1)^2}=\sqrt{9+(-5)^2}=\sqrt{9+25}=\sqrt{34}$


13. A=$(1, -1), B=(3, 3), C=(0, -3)$

\textbf{\underline {Solution}:}

$\overline{AB} =\sqrt{(3-1)^2+(3-(-1))^2}=\sqrt{(2)^2+(3+1)^2}=\sqrt{(2)^2+(4)^2}=\sqrt{4+16}=\sqrt{20}=\sqrt{4}\sqrt{5}=2\sqrt{5}$

$\overline{BC} =\sqrt{(0-3)^2+(-3-3)^2}=\sqrt{(-3)^2+(-6)^2}=\sqrt{9+36}=\sqrt{45}=\sqrt{9}\sqrt{5}=3\sqrt{5}$

$\overline{AC} =\sqrt{(0-1)^2+(-3-(-1))^2}=\sqrt{(-1)^2+(-3+1)^2}=\sqrt{1+(-2)^2}=\sqrt{1+4}=\sqrt{5}$


14. A=$(1, \sqrt{2}), B=(4, 3\sqrt{2}), C=(10, 6\sqrt{2})$

\textbf{\underline {Solution}:}

$\overline{AB} =\sqrt{(4-1)^2+(3\sqrt{2}-\sqrt{2})^2}=\sqrt{(3)^2+(2\sqrt{2})^2}=\sqrt{9+(2)^2)(\sqrt{2})^2}=\sqrt{9+(4)(2)}=\sqrt{9+8}=\sqrt{17}$

$\overline{BC} =\sqrt{(10-4)^2+(6\sqrt{2}-3\sqrt{2})^2}=\sqrt{(6)^2+(3\sqrt{2})^2}=\sqrt{36+(3)^2(\sqrt{2})^2}=\sqrt{36+(9)(2)}=\sqrt{36+18}=\sqrt{54}$

$\overline{AC} =\sqrt{(10-1)^2+(6\sqrt{2}-3\sqrt{2})^2}=\sqrt{(9)^2+(3\sqrt{2})^2}=\sqrt{81+(3)^2(\sqrt{2})^2}=\sqrt{81+(9)(2)}=\sqrt{81+18}=\sqrt{99}=\sqrt{9}\sqrt{11}=3\sqrt{11}$


\textit{In Problems 15-18, determine whether the three given points are the vertices of a right triangle}
\begin{quote}
{}[Given 3 points A, B, C, these points create a triangle if $\overline{AC} < \overline{AB}+\overline{BC}$]
\end{quote}

15. $A=(0, 2), B=(-2, 4), C=(1, 3)$
\textbf{\underline {Solution}:}

$\overline{AB} =\sqrt{(-2-0)^2+(4-2)^2}=\sqrt{(-2)^2+(2)^2}=\sqrt{4+4}=\sqrt{8}=\sqrt{4}\sqrt{2}=2\sqrt{2}$

$\overline{BC} =\sqrt{(1-(-2))^2+(3-4)^2}=\sqrt{(1+2)^2+(-1)^2}=\sqrt{(3)^2+1}=\sqrt{9+1}=\sqrt{10}$

$\overline{AC} =\sqrt{(1-0)^2+(3-2)^2}=\sqrt{(1)^2+(1)^2}=\sqrt{1+1}=\sqrt{2}$

\color{amethyst}Answer: \color{green}{Right Triangle}\color{black}

16. $A=(-1, 3), B=(4, 6), C=(-3, 1)$

\textbf{\underline {Solution}:}

$\overline{AB} =\sqrt{(4-(-1))^2+(6-3)^2}=\sqrt{(4+1)^2+(3)^2}=\sqrt{(5)^2+9}=\sqrt{25+9}=\sqrt{34}$

$\overline{BC} =\sqrt{(-3-4)^2+(1-6)^2}=\sqrt{(-7)^2+(-5)^2}=\sqrt{49+25}=\sqrt{74}$

$\overline{AC} =\sqrt{(1-0)^2+(3-2)^2}=\sqrt{(1)^2+(1)^2}=\sqrt{1+1}=\sqrt{2}$

17. $A=(9, 6), B=(-5, 4), C=(7, 10)$

\textbf{\underline {Solution}:}

$\overline{AB} =\sqrt{(-5-9)^2+(4-6)^2}=\sqrt{(-14)^2+(-2)^2}=\sqrt{196+4}=\sqrt{200}=\sqrt{10}\sqrt{2}$

$\overline{BC} =\sqrt{(7-(-5))^2+(10-4)^2}=\sqrt{(7+12)^2+(6)^2}=\sqrt{19^2+36}=\sqrt{361+36}=\sqrt{397}$

$\overline{AC} =\sqrt{(7-9)^2+(10-6)^2}=\sqrt{(2)^2+(4)^2}=\sqrt{4+16}=\sqrt{20}=\sqrt{4}\sqrt{5}=2\sqrt{5}$

\color{amethyst}Answer: \color{green}{Right Triangle}\color{black}

18. $A=(9, -2), B=(8, 0), C=(-6, -7)$

\textbf{\underline {Solution}:}

$\overline{AB} =\sqrt{(8-9)^2+(0-(-2))^2}=\sqrt{(-1)^2+(0+2)^2}=\sqrt{1+(2)^2}=\sqrt{1+4}=\sqrt{5}$

$\overline{BC} =\sqrt{(-6-8)^2+(-7-0)^2}=\sqrt{(-14)^2+(-7)^2}=\sqrt{196+49}=\sqrt{245}=\sqrt{49}\sqrt{5}=7\sqrt{5}$

$\overline{AC} =\sqrt{(-6-9)^2+(-7-(-2))^2}=\sqrt{(-15)^2+(-7+2)^2}=\sqrt{225+(-5)^2}=$

$\sqrt{225+25}=\sqrt{250}=\sqrt{25}\sqrt{10}=5\sqrt{10}=5\sqrt{2}\sqrt{5}$

\medskip
19. $P_{1} = (1, 5), P_{2} = (x, 2), \overline{P_1P_2} = 5$

\smallskip
\textbf{\underline {Solution}:}
\smallskip

$5=\sqrt{(x-1)^2+(2-5)^2}=\sqrt{(x-1)(x-1)+(-3)^2}=\sqrt{x^2-2x+1+9}=\sqrt{x^2-2x+10}$

$25=x^2-2x+10$

$0=x^2-2x+10-25$

$0=x^2-2x-15$

$0=(x+3)(x-15)$

$0=(x+3)$ or $0=(x-15)$

$\therefore -3=x$ or $15=x$

\medskip
20. $P_{1} = (-3, y), P_{2} = (9, 2), \overline{P_1P_2 }= 13$

\smallskip
\textbf{\underline {Solution}:}
\smallskip

$13=\sqrt{(9-(-3))^2+(2-y)^2} = \sqrt{(9+3)^2+(2-y)(2-y)}$

$13^2=(\sqrt{(9-(-3))^2+(2-y)^2})^2) = (\sqrt{(12)^2+(2-y)(2-y)})^2$

$169=144+4-2y-2y+y^2$

$169=144+4-4y+y^2=y^2-4y+148$

$169-169=144+4-4y+y^2=y^2-4y+148-169$

$0=144+4-4y+y^2=y^2-4y-21$

$0=(x+3)(x-7)$

$\therefore x=-3$, or $x=7$

\medskip
21. $P_{1} = (x, x), P_{2} = (1, 4), \overline{P_1P_2} = \sqrt{5}$

\smallskip
\textbf{\underline {Solution}:}\\
\smallskip

$\sqrt{(1-x)^2+(4-x)^2}=\sqrt{5} \implies (\sqrt{(1-x)^2+(4-x)^2})^2=(\sqrt{5})^2 \implies \sqrt{(1-x)^2+(4-x)^2}=5$

$\therefore\,(1-x)(1-x)+(4-x)(4-x)=5$

$1-x-x+x^2+16-4x-4x+x^2=5$

$1-2x+x^2+16-8x+x^2=5$

$1-2x+2x^2+16-8x=5$

$2x^2-10x+17=5$

$2x^2-10x+17-5=5-5$

$2x^2-10x+12=0$

$\frac{1}{2}[2x^2-10x+12=0]$

$x^2-5x+6=0$

$(x-2)(x-3))=0$

$\therefore\,x=2$ or $x=3$

\medskip
22. $P_{1} = (x, 2x), P_{2} = (2x, 1), \overline{P_1P_2} = \sqrt{2}$

\smallskip
\textbf{\underline {Solution}:}

$\sqrt{2}=\sqrt{(2x-x)^2+(1-2x)^2}$

$(\sqrt{2})^2=( \sqrt{(2x-x)^2+(1-2x)^2} )^2$

$2=(2x-x)^2+(1-2x)^2$

$2=(2x-x)(2x-x)+(1-2x)(1-2x)$

$2=4x^2-2x^2-2x^2+x^2+1-2x-2x+4x^2$

$2=4x^2-4x^2+x^2+1-4x+4x^2$

$2=5x^2-4x+1$

$2\color{green}-2\color{black}=5x^2-4x+1\color{green}-2\color{black}$

$0=5x^2-4x-1$ 

$0=(5x+1)(x-1) \implies\, x=-\frac{1}{5}$ or $x=1$

\medskip
23. Show that $(5,2)$ is on the perpendicular bisector of the segment $\overline{AB}$ where $A = (1,3)$ and $B=(4, -2)$

\smallskip
\textbf{\underline {Solution}:}

A bisector divides in half and a perpendicular line has slope 0, therefore a perpendicular bisector is a line of slope 0 that halves the segment $\overline{AB}$.

In this instance the point on the perpendicular bisector will participate in creating a triangular relationship between all three points.

Therefore if we set $C=(5,2)$. Then the three points make a triangle if $\overline{AC} <\overline{AB}+\overline{BC}$

$\overline{AB}=\sqrt{(4-1)^2+(-2-3)^2}=\sqrt{3^2+(-5)^2}=\sqrt{9+25}=\sqrt{34}$

$\overline{BC}=\sqrt{(5-4)^2+(2-(-2))^2}=\sqrt{1^2+(2+2)^2}=\sqrt{1+16}=\sqrt{17}$

$\overline{AC}=\sqrt{(5-1)^2+(2-3)^2}=\sqrt{4^2+(-1)^2}=\sqrt{16+1}=\sqrt{17}$

Since $\sqrt{17} < \sqrt{34}+\sqrt{17}$, $C=(5,2)$ is proven to be on the perpendicular bisector of $\overline{AB}$. [Q.E.D.]

\medskip
24. Show that $(-2,4), (2,0), (2,8)$ and $(6,4)$ are the vertices of a square.

\smallskip
\textbf{\color{blue}\underline {Solution}:}

\medskip
25. Show that $(1,1), (4,1), (3,-2)$ and $(0,-2)$ are the vertices of a parallelogram.

\smallskip
\textbf{\color{blue}\underline {Solution}:}

\medskip
26. Find all possible values for $y$ so that $(5,8), (-4,11)$ and $(2,y)$ are the vertices of a right triangle.

\smallskip
\textbf{\color{blue}\underline {Solution}:}

\medskip
27. Determine whether each of the following points is inside, on or outside the circle with center $(-2, 3)$ and radius 5: $(1,7), (-3,8), (2,0), (-5,7), (0,-1), (-5,-1), (-6,6), (4,2)$

\smallskip
\textbf{\color{blue}\underline {Solution}:}

\bigskip
\textbf{C}

\medskip
28. Find the center and radius of the circle circumscribed about the triangle with vertices $(5,1)$, $(6,0)$ and $(-1, -7)$


\medskip
29. Show that a triangle with vertices $(x_1, y_1)$, $(x_2,y_2)$, $(x_3, y_3)$ has area 

\begin{align}
\frac{1}{2}
\begin{vmatrix}
x_1y_2 +x_2y_3 +x_3y_1 - x_1y_3 -x_2y_1 -x_3y_2
\end{vmatrix}
= 
 \begin{vmatrix}
 \frac{1}{2}
 \begin{vmatrix}
 x_1 & y_1 & 1\\
 x_2 & y_2 & 1\\
 x_3 & y_3 & 1\\
 \end{vmatrix}
 \end{vmatrix}
\end{align}

{}[Hint: Consider the rectangle with sides parallel to the coordinate axes and containing the vertices of the triangle.]

30. Prove analytically that if the diagonals of a parallelogram are equal, then the parallelogram is a rectangle.

{}[ Hint: Place the axes as shown in fig 1.10 and show that $\overline{AC} = \overline{BD}$ implies that $A$ is the origin]

31. Prove analytically that the sum of the lengths of two sides of a triangle is greater than the length of the third side.

\end{document}