% !TeX root = section-1.1.tex
\documentclass[10pt,letterpaper]{article}
\usepackage{mathtools}
\usepackage{amsmath}
\usepackage{amsfonts}
\usepackage{amssymb}
\usepackage{mathrsfs}
\usepackage{graphicx}
\usepackage{caption}
\usepackage{array, xcolor}
\usepackage[inline]{enumitem}
\usepackage[left=2cm,right=2cm,top=2cm,bottom=2cm]{geometry}

\begin{document}
Exercises 1 through 4 concern the binary operation $ \ast $ defined on $S=\{a,b,c,d,e\}$ by means of Table 1.4.
  \begin{table}[h]
    \begin{center}
      \begin{tabular}{l!{\vrule width 2pt}l|l|l|l|l}
        $\ast$ & a & b & c & d & e \\
        \noalign{\hrule height 2pt}
        a & a & b & c & b & d \\
        \hline
        b & b & c & a & e & c\\
        \hline
        c & c & a & b & b & a\\
        \hline
        d & b & c & b & e & d \\
        \hline
        e & d & b & a & d & c
      \end{tabular}
    \end{center}
    \captionsetup{labelformat=empty}
    \caption{Table 1.4}
  \end{table}
  \begin{enumerate}
    \item Compute $ b \ast c $, $ c \ast c $, $ [(a \ast c) \ast e] \ast a $
    \item Compute $ (a \ast b) \ast c $ and $a \ast (b \ast c)$. Can you say on the basis of this computation whether $\ast$ is associative?
    \item Compute $(b \ast d) \ast c$ and $b \ast (d \ast c)$. Can you say on the basis of this computation whether $\ast$ is associative?
    \item Is * commutative? Why?
  \end{enumerate}
  
  \begin{enumerate}[resume]
    \item Complete Table 1.5 so as to define a commutative binary operation $\ast $ on $S=\{a,b,c,d\}$. \\
    \begin{table}[h]
      \begin{center}
        \captionsetup{labelformat=empty}
        \caption{Table 1.5}
        \begin{tabular}{l!{\vrule width 2pt}l|l|l|l} 
          $\ast$ & a & b & c & d \\
          \noalign{\hrule height 2pt}
          a & a & b & c &  \\
          \hline
          b & b & d &  & c \\
          \hline
          c & c & a & d & b \\
          \hline
          d & b &  &  & a
        \end{tabular}
      \end{center}
    \end{table}
    \item Complete Table 1.6 so as to define a associative binary operation $\ast $ on $S=\{a,b,c,d\}$. 
          Assume this is possible and compute the missing entries\\
    \begin{table}[h]
      \begin{center}
        \captionsetup{labelformat=empty}
        \caption{Table 1.6}
        \begin{tabular}{l!{\vrule width 2pt}l|l|l|l} 
          $\ast$ & a & b & c & d \\
          \noalign{\hrule height 2pt}
          a & a & b & c & d \\
          \hline
          b & b & a & c & d \\
          \hline
          c & c & d & c & d \\
          \hline
          d &  &  &  & 
        \end{tabular}
      \end{center}
    \end{table}
  \end{enumerate}

  In Exercises 7 through 11, determine whether the binary operator $\ast$ defined is commutative and whether $\ast$ is associative.
  \begin{enumerate}[resume]
    \item $\ast$ defined on $\mathbb{Z}$ by $a \ast b = a-b$
    \item $\ast$ defined on $\mathbb{Q}$ by $a \ast b = ab+1$
    \item $\ast$ defined on $\mathbb{Q}$ by $a \ast b = ab/2$
    \item $\ast$ defined on $\mathbb{Z}^+$ by $a \ast b = 2^{ab}$
    \item $\ast$ defined on $\mathbb{Z}^+$ by $a \ast b = a^b$
  \end{enumerate}

  \begin{enumerate}[resume]
    \item Let $S$ be a set having exactly one element. How many different binary operations
    can be defined on $S$? Answer the question if $S$ has exactly 2 elements ; exactly 3 elements; exacly $n$ elements.

    \item How many different commutative binary operations can be defined on a set of 2 elements? \\
    on a set of 3 elements? on a set of $n$ elements.
  \end{enumerate}

  \section*{CONCEPTS}

  In exercises 14 through 19, determine whether the definition of $\ast$ does give a binary operation \\
  on the set. In the event that $\ast$ is not a binary operation. state whether Condition 1, Condition 2\\
  or both of these conditions on page 14 are violated.
  \begin{enumerate}[resume]
    \item On $\mathbb{Z}^+$, define $\ast$ by $a*b=a-b$
    \item On $\mathbb{Z}^+$, define $\ast$ by $a*b=a^b$
    \item On $\mathbb{R}$, define $\ast$ by $a*b=a-b$
    \item On $\mathbb{Z}^+$, define $\ast$ by $a*b=c$ where $c$ is the smallest integer greater than both a and b.
    \item On $\mathbb{Z}^+$, define $\ast$ by $a*b=c$ where $c$ is at least 5 more than $a+b$.
    \item On $\mathbb{Z}^+$, define $\ast$ by $a*b=c$ where $c$ is the largest integer less than the product of $a$ and $b$.
  \end{enumerate}

  \begin{enumerate}[resume]
    \item Mark each of the following true or false.
    \begin{enumerate}
      \item If $\ast$ is any binary operation on $S$, then $a\ast a = a$ for all $a \in S$.
      \item If $\ast$ is any commutative binary operation on any set $S$, then $a \ast (b \ast c) = (b \ast c) \ast a$ for all $a, b,c \in S$.
      \item If $\ast$ is any associative binary operation on any set $S$, then $a \ast (b \ast c)=(b \ast c) \ast a$ for all $a, b, c \in S$
      \item The only binary operations of any importance are those defined on sets of numbers.
      \item A binary operation $\ast$ on a set $S$ is  commutative if there exist $a, b \in S$ such that $a\ast b = b \ast a$.
      \item Every binary operation defined on a set $S$ having exactly one element is both commutative and associative.
      \item A binary operation on a set $S$ assigns at least one element of $S$ to some ordered pair of elements of $S$.
      \item A binary operation on a set $S$ assigns at most one element of $S$ to some ordered pair of elements of $S$.
      \item A binary operation on a set $S$ assigns exactly one element of $S$ to some ordered pair of elements of $S$.
      \item A binary operation on a set $S$ may assign more than one element of $S$ to some ordered pair of elements of $S$.
    \end{enumerate}
  \end{enumerate}

  \begin{enumerate}[resume]
    \item Give a set different from any of those described in the examples of the text and not a set of numbers. Define two different binary
    operations $\ast$ and $\ast$ on this set. Be sure that your set is \textit{well defined}.
  \end{enumerate}

  \section*{THEORY}

  \begin{enumerate}[resume]
    \item Prove that if $*$ is an associative and commutative binary operation on a set $S$, then 
    \[ (a\ast b)\ast (c \ast d)=[(d \ast c) \ast a] \ast b \] 
    for all $a, b,c,d \in S$. Assume the associative law only for triples as in the definition,
    that is, assume only \[ (a * y)*z= x*(y*z) \] for all $x, y,z \in S$.
  \end{enumerate}

  \noindent In exercises 23 and 24, either prove the statement or give a counterexample.
  \begin{enumerate}[resume]
    \item Every binary operation on a set consisting of a single element is both commutative and associative.
    \item Every commutative binary operation on a set having just two elements is associative.
  \end{enumerate}

  \noindent Let $F$ be the set of all real-valued functions with domain $\mathbb{R}$. Function composition $\circ$
  $F$ is defined by \[ (f \circ g)(x) = f(g(x)) \ \ \ \ \text{ for all \ \ $x \in \mathbb{R}$}  \] 
  In Exercises 25 through 29, either prove the statement or give a counterexample.
  \begin{enumerate}[resume]
    \item Function composition on $F$ is commutative.
    \item Function composition on $F$ is associative.
    \item Function multiplication on $F$ is commutative.
    \item Function addition on $F$ is commutative.
    \item If $*$ and $*^{\prime}$ are any two binary operations on a set $S$, then
    \[ a*(b*^{\prime} c) = (a*b) *^{\prime} (a*c) \ \ \  \text{ for all $a, b, c \in S$}\]
  \end{enumerate}

  \begin{enumerate}[resume]
    \item Observe that the binary operations $*$ and $*^{\prime}$ on the set $\{a, b\}$ given by the tables
    \begin{center}    
    \begin{tabular}{ccc}%
      \begin{minipage}{.2\linewidth}
      \begin{tabular}{c!{\vrule width 2pt}c|c}
        $*$ & a & b \\
        \noalign{\hrule height 2pt}
        a & a & a \\
        \hline
        b & a & b \\
      \end{tabular} 
      \end{minipage} &

      \begin{minipage}{.2\linewidth}
      and 
      \end{minipage} &

      \begin{minipage}{.2\linewidth}
        \begin{tabular}{c!{\vrule width 2pt}c|c}
          $*^{\prime}$ & a & b \\
          \noalign{\hrule height 2pt}
          a & a & a \\
          \hline
          b & b & b \\
        \end{tabular}
        \end{minipage}
    \end{tabular}
  \end{center}
    provide the \textit{same type of algebraic structure} on $\{a, b\}$, 
    in the sense that if the table for $*^{\prime}$ is rewritten as

    \begin{center}
      \begin{tabular}{c!{\vrule width 2pt}c|c}
        $*^{\prime}$ & b & a \\
        \noalign{\hrule height 2pt}
        b & b & b \\
        \hline
        a & b & a \\
      \end{tabular}
    \end{center}

    this table for $*^{\prime}$ looks just like that for $*$, with the roles of $a$ and $b$ interchanged.

    \begin{enumerate}
      \item try to give a natural definition of a concept of two binary operations $*$ and $*^{\prime}$ on the same set 
      giving \textit{algebraic structures of the same type}, which generalizes the observation.
      \item How many different types of algebraic structures are given by the 16 possible different binary operations on a set of 2 elements.
    \end{enumerate}    

  \end{enumerate}




\end{document}