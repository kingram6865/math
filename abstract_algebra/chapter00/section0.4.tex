% !TeX root = section0.4.tex
\documentclass[10pt,letterpaper]{article}
\usepackage{mathtools}
\usepackage{amsmath}
\usepackage{amsfonts}
\usepackage{amssymb}
\usepackage{mathrsfs}
\usepackage{graphicx}
\usepackage[inline]{enumitem}
\usepackage[left=2cm,right=2cm,top=2cm,bottom=2cm]{geometry}

\begin{document}
In exercises 1 through 19, compute the given arithmetic expression and give the answer in the form $a+bi$ for $a,b\in \mathbb{R}$
  \begin{enumerate}
    \item $(2-3i) + (4+5i)$ 
    \item $i + (5i-3)$
    \item $(5+7i) - (3-2i)$
    \item $(1-3i) - (-4+2i)$
    \item $i^3$
    \item $i^4$
    \item $i^{23}$
    \item $(-i)^{35}$
    \item $(4-i)(5+3i)$
    \item $(8+2i)(3-i)$
    \item $(2-3i)(4+i) + (6-5i)$
    \item $(1+i)^3$
    \item $(1-i)^5$ (Use the binomial theorem)
    \item $\dfrac{7-5i}{1+6i}$
    \item $\dfrac{i}{1+i}$
    \item $\dfrac{1-i}{i}$
    \item $\dfrac{i(3+i)}{2-4i}$ 
    \item $\dfrac{3+7i}{(1+i)(2-3i)}$
    \item $\dfrac{(1-i)(2+i)}{(1-2i)(1+i)}$
    \item $\lvert3-4i\rvert$
    \item $\lvert6+4i\rvert$
  \end{enumerate}

In Exercises 22 through 25, write the given complex number $z$ in the polar form $\lvert z \rvert (p+qi) $ where $\lvert p+qi \rvert =1$
  \begin{enumerate}[resume]
    \item $3-4i$
    \item $-1+i$
    \item $12+5i$
    \item $-3+5i$
  \end{enumerate}

  \begin{enumerate}[resume]
    \item Let $z_1$ and $z_2$ be complex numbers with $z_2 \ne 0$. Describe the geometric meaning of $z_1/z_2$ in terms of norms and polar angles.
  \end{enumerate}
  \pagebreak
In Exercises 27 through 32, find all complex solutions of the given equation.
  \begin{enumerate}[resume]
    \item $z^4=1$
    \item $z^4=-1$
    \item $z^3=-8$
    \item $z^3=-27i$
    \item $z^6=1$
    \item $z^6=64$
  \end{enumerate}

  In Exercises 33 through 40, compute the given arithmetic matrix expression, if it is defined.
  \begin{enumerate}[resume]
    \item $\begin{bmatrix*}[r] -2 & 4 \\ 1 & 5  \end{bmatrix*} + \begin{bmatrix*}[r] 4 & -3 \\ 1 & 2  \end{bmatrix*}$
    \item $\begin{bmatrix*}[c] 1+i & -2 & 3-i \\ 4 & i & 2-i  \end{bmatrix*} + \begin{bmatrix*}[c] 3 & i-1 & -2+i \\ 3-i & 1+i & 0  \end{bmatrix*}$
    \item $\begin{bmatrix*}[r] 1 & -1 \\ 4 & 1 \\ 3 & -2i \end{bmatrix*} - \begin{bmatrix*}[c] 3-i & 4i \\ 2 & 1+i \\ 3 & -i \end{bmatrix*}$
    \item $\begin{bmatrix*}[r] 1 & -1 \\ 3 & 2  \end{bmatrix*} \begin{bmatrix*}[r] 2 & 4 \\ -1 & 3  \end{bmatrix*}$
    \item $\begin{bmatrix*}[r] 3 & 1 \\ -4 & 2  \end{bmatrix*} \begin{bmatrix*}[r] 1 & 5 & -3 \\ 2 & 1 & 6  \end{bmatrix*}$
    \item $\begin{bmatrix*}[r] 4 & -1 \\ 1 & 2  \end{bmatrix*} \begin{bmatrix*}[r] 1 & 0 \\ -1 & 7 \\ 3 & 1 \end{bmatrix*}$
    \item $\begin{bmatrix*}[r] i & 1 \\ -2 & 1  \end{bmatrix*} \begin{bmatrix*}[r] 3i & 1 \\ 4 & -2i \end{bmatrix*}$
    \item $\begin{bmatrix*}[r] 1 & -1 \\ 1 & 0 \end{bmatrix*}^4 $
    \item $\begin{bmatrix*}[r] 1 & -i \\ i & 1 \end{bmatrix*}^4 $
  \end{enumerate}

  \begin{enumerate}[resume]
    \item Give an example in $M_2(\mathbb{Z})$ showing that matrix multiplication is not commutative.
  \end{enumerate}

  \begin{enumerate}[resume]
    \item By Experimentation, find $\begin{bmatrix*}[r] 0 & 1 \\ -1 & 0 \end{bmatrix*}^{-1}$.
  \end{enumerate}

  \begin{enumerate}[resume]
    \item By Experimentation, find $\begin{bmatrix*}[r] 2 & 0 & 0 \\ 0 & 4 & 0 \\ 0 & 0 & -1 \end{bmatrix*}^{-1}$.
  \end{enumerate}

  \begin{enumerate}[resume]
    \item Prove that if $A, B \in M_4(\mathbb{C})$ are invertible then $AB$ and $BA$ are invertible also.
  \end{enumerate}

  \begin{enumerate}[resume]
    \item Let $z$. Prove, using mathematical induction, that\mbox{}\\
          $z^n=r^n(\cos n\theta + \sin n\theta)$ for all $n \in \mathbb{Z}$
  \end{enumerate}


\end{document}