% !TeX root = section0.2.tex
\documentclass[10pt,letterpaper]{article}
\usepackage{mathtools}
%\usepackage{xcolor}
%\usepackage[utf8]{inputenc}
\usepackage{amsmath}
\usepackage{amsfonts}
\usepackage{amssymb}
\usepackage{mathrsfs}
\usepackage{graphicx}
\usepackage[inline]{enumitem}
\usepackage[left=2cm,right=2cm,top=2cm,bottom=2cm]{geometry}
%\definecolor{amethyst}{rgb}{0.6, 0.4, 0.8}
\title{A First Course in Abstract Algebra, Fraleigh, 4th Edition 1989\\ Chapter 0 - A Few Preliminaries\\Section 0.2: Sets and Equivalence Relations\\ Exercises 0.2}
\begin{document}
\maketitle
  For exercises 1 through 4, describe the set by listing its elements. 
  \begin{enumerate}
    \item $\{x \in \mathbb{R} \mid x^2 = 3\}$
    \item $\{m \in \mathbb{Z} \mid m^2 = 3\}$
    \item $\{m \in \mathbb{Z} \mid mn = 60 \text{ for some } n \in \mathbb{Z}\}$
    \item $\{m \in \mathbb{Z} \mid m^2 - m < 115\}$
  \end{enumerate}

  In exercises 5 through 10, decide whether the object described is indeed a set (is well defined). 
  Give an alternate description of each set.
  \begin{enumerate}[resume]
    %\setcounter{enumi}{4}
    \item $\{n \in \mathbb{Z}^+ \mid n \text{ is a large number}\}$
    \item $\{n \in \mathbb{Z} \mid n^2 < 0\}$
    \item $\{n \in \mathbb{Z}^+ \mid 39 < n^3 < 57\}$
    \item $\{x \in \mathbb{Q}^+ \mid \text{ the denominator of $x$ is greater than 100}\}$
    \item $\{x \in \mathbb{Q}^+ \mid \text{$x$ may be written with denominator greater than 100}\}$
    \item $\{x \in \mathbb{Q}^+ \mid \text{$x$ may be written with denominator less than 3}\}$
  \end{enumerate}

  In exercises 11 through 17, determine whether the given relation is an equivalence relation on the set. Describe the partition arising from each equivalence relation.
  \begin{enumerate}[resume]
    %\setcounter{enumi}{10}
    \item $n\: \mathscr{R}\: m \in \mathbb{Z} \text{ if } nm > 0$
    \item $x\: \mathscr{R}\: y \in \mathbb{R} \text{ if } x \ge y$
    \item $x\: \mathscr{R}\: y \in \mathbb{R} \text{ if } |x| = |y|$
    \item $x\: \mathscr{R}\: y \in \mathbb{R} \text{ if } |x-y| \le 3$
    \item $n\: \mathscr{R}\: m \text{ in } \mathbb{Z}^+ \text{ if $n$ and $m$ have the same number of digits in the usual base ten notation }$
    \item $n\: \mathscr{R}\: m \text{ in } \mathbb{Z}^+ \text{ if $n$ and $m$ have the same number of digits in the usual base ten notation }$
    \item $n\: \mathscr{R}\: m \text{ in } \mathbb{Z}^+ \text{ if $n$ and $m$ have the same number of digits in the usual base ten notation }$
  \end{enumerate}

  \begin{enumerate}[resume]
    \item Let $n$ be a particular integer in $\mathbb{Z}^+$. Show that congruence modulo $n$ is an equivalence relation on ${Z}$.
  \end{enumerate}

  In exercises 19 through 23, describe all residue classes of $\mathbb{Z}$ modulo $n$ is an equivalence relation on $\mathbb{Z}$

  \begin{enumerate*}[resume]
    \setcounter{enumi}{18}
    \item $n = 1$
    \item $n = 2$
    \item $n = 3$
    \item $n = 4$
    \item $n = 8$
  \end{enumerate*}

  \begin{enumerate*}[resume]
    \item 1 element
    \item 2 elements
    \item 3 elements
    \item 4 elements
    \item 5 elements
  \end{enumerate*}

\end{document}